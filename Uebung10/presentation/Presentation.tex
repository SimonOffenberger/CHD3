% ======================================================================
%  Code Review Presentation Template – CHD3 (University)
%  Author: Simon Offenberger
%  Date: \today
% ======================================================================

\documentclass[aspectratio=169,11pt]{beamer}

% =======================
%   THEMES & PACKAGES
% =======================
\usetheme{metropolis} % clean, professional theme
\usepackage[utf8]{inputenc}
\usepackage[T1]{fontenc}
\usepackage[ngerman]{babel}
\usepackage{graphicx}
\usepackage{booktabs}
\usepackage{xcolor}
\usepackage{listings}
\usepackage{pdfpages}


% =======================
%   DESIGN SETTINGS
% =======================
\definecolor{uniBlue}{RGB}{0,68,136}
\definecolor{lightGray}{RGB}{245,245,245}
\setbeamercolor{frametitle}{bg=uniBlue, fg=white}
\setbeamercolor{title}{fg=uniBlue}
\setbeamercolor{progress bar}{use=alerted text, fg=uniBlue}

% =======================
%   CODE LISTINGS STYLE
% =======================
\lstdefinelanguage{VHDL}{
  morekeywords=[1]{
    architecture,begin,block,body,buffer,bus,case,component,configuration,
    constant,disconnect,downto,else,elsif,end,entity,exit,file,for,function,
    generate,generic,group,guarded,if,in,inertial,inout,is,label,library,linkage,
    literal,loop,map,mod,nand,new,next,nor,not,null,of,on,open,or,others,out,
    package,port,postponed,procedure,process,pure,range,record,register,reject,
    rem,report,return,rol,ror,select,severity,signal,shared,sla,sll,sra,srl,
    subtype,then,to,transport,type,unaffected,units,until,use,variable,wait,
    when,while,with,xnor,xor
  },
  morekeywords=[2]{std_logic,std_logic_vector,signed,unsigned,integer,boolean,natural,positive,time,real,bit,bit_vector},
  morekeywords=[3]{ieee,std,work,math_real,numeric_std,std_logic_1164},
  sensitive=false, % VHDL is case-insensitive
  morecomment=[l]--,
  morecomment=[s]{/*}{*/}, % optional (non-standard in VHDL, but sometimes used in docs)
  morestring=[b]",
  alsoletter={._}, % treat dot/underscore as part of identifiers (ieee.std_logic_1164)
}

\lstdefinestyle{vhdl}{
  language=VHDL,
  basicstyle=\ttfamily\tiny,
  columns=fullflexible,
  keepspaces=true,
  showstringspaces=false,
  breaklines=true,
  tabsize=2,
  commentstyle=\itshape\color{gray!70},
  stringstyle=\color{orange!70!black},
  keywordstyle=[1]\bfseries\color{blue!70!black},
  keywordstyle=[2]\bfseries\color{teal!70!black},
  keywordstyle=[3]\bfseries\color{purple!70!black},
  numbers=left,
  numberstyle=\tiny\color{gray!70},
  numbersep=8pt,
  frame=single,
  rulecolor=\color{black!20},
  captionpos=b
}

% =======================
%   TITLE INFORMATION
% =======================
\title[CHD3 Übung 10]{Übung 10, Aufgabe 2}
\subtitle{Reaktionszeit-Spiel fuer zwei Personen} 
\author{Simon Offenberger S2410306027@fhooe.at} 
\institute{FH Hagenberg}
\date{\today}

% =======================
%   FOOTLINE ANPASSUNG
% =======================
\setbeamertemplate{footline}{
  \leavevmode%
  \hbox{%
    \begin{beamercolorbox}[wd=.8\paperwidth,ht=2.5ex,dp=1.5ex,leftskip=1em]{author in head/foot}%
      \usebeamerfont{author in head/foot}\insertshortauthor
    \end{beamercolorbox}%
    \begin{beamercolorbox}[wd=.2\paperwidth,ht=2.5ex,dp=1.5ex,rightskip=1em plus1fil]{date in head/foot}%
      \usebeamerfont{date in head/foot} \hfill  \insertframenumber{} / \inserttotalframenumber
    \end{beamercolorbox}%
  }%
  \vskip0pt%
}

\setbeamertemplate{navigation symbols}{} % entfernt die Navigationssymbole

% =======================
%   DOCUMENT
% =======================
\begin{document}

% -----------------------
\begin{frame}
  \titlepage
\end{frame}

% -----------------------
\begin{frame}{Agenda}
    \begin{itemize}
      \item Decoupling - Spannungsstabilisierung
      \item GPIO - Schutzbeschaltung
      \item IR-Emitter
      \item Spannungsversorgung
  \end{itemize}
\end{frame}

% -----------------------
\begin{frame}{Finite State Machine  Moore}
\includegraphics[width= 11 cm]{./../doc/images/Aufgabe2/FSM.png}
\end{frame}
% -----------------------

% -----------------------
\begin{frame}[fragile]{State Register}
\begin{lstlisting}[style=vhdl]
  -- State Register
  process (iClk, inResetAsync) is
  begin
    if (inResetAsync = not('1')) then
      State <= Locked;
    elsif (rising_edge(iClk)) then
      if(iEnable = '0') then
        State <= State; -- hold state when not enabled
      else
        State <= NextState;
      end if;
    end if;
  end process;
\end{lstlisting}
\end{frame}
% -----------------------

% -----------------------
\begin{frame}[fragile]{Next State Logic}
\begin{columns}[T,onlytextwidth]
  \column{0.49\textwidth}
  \scriptsize
  \begin{lstlisting}[style=vhdl]
-- State Transition Process
NextStateLogic : process (State,iEnable,iA_Sync,iB_Sync) is
begin

  NextState <= State;    -- default hold state
  oLeds <= cLEDOFF;      -- default all leds off
  oZero <= '0';          -- default zero off
  oEnableCounter <= '0'; -- default counter disabled

  case State is
    when Locked => 
      if(iA_Sync = '1') then
        NextState <= Unlocked;
      end if;
      oZero <= '1';                    -- reset counters
      oLeds(cLED_LOCKED_INDEX) <= '1'; -- indicate locked state
      
    when Unlocked =>
      if(iB_Sync = '1') then
        NextState <= CountUpTime;
      end if;
      oLeds(cLED_UNLOCKED_INDEX) <= '1'; -- indicate locked state
  \end{lstlisting}

  \column{0.49\textwidth}
  \scriptsize
  \begin{lstlisting}[style=vhdl]
    when CountUpTime =>
      oEnableCounter <= '1'; -- enable counter
      if(iA_Sync = '1') then
        NextState <= ShowResult;
      end if;
      oLeds(cLED_COUNTUP_INDEX) <= '1'; -- indicate locked state
      
    when ShowResult =>
      if(iB_Sync = '1') then
        NextState <= Locked;
      end if;
      oLeds(cLED_SHOWRESULT_INDEX) <= '1'; -- indicate locked state
      
    when others =>
      NextState <= cStateAllOff;
  end case;

end process;
  \end{lstlisting}
\end{columns}
\end{frame}
% -----------------------


% -----------------------
\begin{frame}[standout]
  Fragen?
\end{frame}

\end{document}
