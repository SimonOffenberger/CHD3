% ======================================================================
%  Code Review Presentation Template – CHD3 (University)
%  Author: Simon Offenberger
%  Date: \today
% ======================================================================

\documentclass[aspectratio=169,11pt]{beamer}

% =======================
%   THEMES & PACKAGES
% =======================
\usetheme{metropolis} % clean, professional theme
\usepackage[utf8]{inputenc}
\usepackage[T1]{fontenc}
\usepackage[ngerman]{babel}
\usepackage{graphicx}
\usepackage{booktabs}
\usepackage{xcolor}
\usepackage{listings}
\usepackage{pdfpages}
\usepackage{tikz}
\usepackage{hyperref}



% =======================
%   DESIGN SETTINGS
% =======================
\definecolor{uniBlue}{RGB}{0,68,136}
\definecolor{lightGray}{RGB}{245,245,245}
\setbeamercolor{frametitle}{bg=uniBlue, fg=white}
\setbeamercolor{title}{fg=uniBlue}
\setbeamercolor{progress bar}{use=alerted text, fg=uniBlue}

% =======================
%   CODE LISTINGS STYLE
% =======================
\lstdefinelanguage{VHDL}{
  morekeywords=[1]{
    architecture,begin,block,body,buffer,bus,case,component,configuration,
    constant,disconnect,downto,else,elsif,end,entity,exit,file,for,function,
    generate,generic,group,guarded,if,in,inertial,inout,is,label,library,linkage,
    literal,loop,map,mod,nand,new,next,nor,not,null,of,on,open,or,others,out,
    package,port,postponed,procedure,process,pure,range,record,register,reject,
    rem,report,return,rol,ror,select,severity,signal,shared,sla,sll,sra,srl,
    subtype,then,to,transport,type,unaffected,units,until,use,variable,wait,
    when,while,with,xnor,xor
  },
  morekeywords=[2]{std_logic,std_logic_vector,signed,unsigned,integer,boolean,natural,positive,time,real,bit,bit_vector},
  morekeywords=[3]{ieee,std,work,math_real,numeric_std,std_logic_1164},
  sensitive=false, % VHDL is case-insensitive
  morecomment=[l]--,
  morecomment=[s]{/*}{*/}, % optional (non-standard in VHDL, but sometimes used in docs)
  morestring=[b]",
  alsoletter={._}, % treat dot/underscore as part of identifiers (ieee.std_logic_1164)
}

\lstdefinestyle{vhdl}{
  language=VHDL,
  basicstyle=\ttfamily\tiny,
  columns=fullflexible,
  keepspaces=true,
  showstringspaces=false,
  breaklines=true,
  tabsize=2,
  commentstyle=\itshape\color{gray!70},
  stringstyle=\color{orange!70!black},
  keywordstyle=[1]\bfseries\color{blue!70!black},
  keywordstyle=[2]\bfseries\color{teal!70!black},
  keywordstyle=[3]\bfseries\color{purple!70!black},
  numbers=left,
  numberstyle=\tiny\color{gray!70},
  numbersep=8pt,
  frame=single,
  rulecolor=\color{black!20},
  captionpos=b
}

% =======================
%   TITLE INFORMATION
% =======================
\title[CHD3 Übung 10]{Übung 10, Aufgabe 2}
\subtitle{Reaktionszeit-Spiel fuer zwei Personen} 
\author{Simon Offenberger S2410306027@fhooe.at} 
\institute{FH Hagenberg}
\date{\today}

% =======================
%   FOOTLINE ANPASSUNG
% =======================
\setbeamertemplate{footline}{
  \leavevmode%
  \hbox{%
    \begin{beamercolorbox}[wd=.8\paperwidth,ht=2.5ex,dp=1.5ex,leftskip=1em]{author in head/foot}%
      \usebeamerfont{author in head/foot}\insertshortauthor
    \end{beamercolorbox}%
    \begin{beamercolorbox}[wd=.2\paperwidth,ht=2.5ex,dp=1.5ex,rightskip=1em plus1fil]{date in head/foot}%
      \usebeamerfont{date in head/foot} \hfill  \insertframenumber{} / \inserttotalframenumber
    \end{beamercolorbox}%
  }%
  \vskip0pt%
}

\setbeamertemplate{navigation symbols}{} % entfernt die Navigationssymbole

% =======================
%   DOCUMENT
% =======================
\begin{document}

% -----------------------
\begin{frame}
  \titlepage
\end{frame}

% -----------------------
\begin{frame}{Agenda}
    \begin{itemize}
      \item Struktur des Reaktionszeit-Spiels
      \item Edge Detection
      \item Finite State Machine
      \item Counter
      \item Hex to 7-Segment Decoder
      \item Simulation Reaction Time Game
      \item Ressource Summary
  \end{itemize}
\end{frame}

% -----------------------
\begin{frame}{Struktur des Reaktionszeit-Spiels}
\includegraphics[width=\textwidth]{./../doc/images/Aufgabe2/Game_Struct_RTL (1).png}
\end{frame}
% -----------------------

% -----------------------
\begin{frame}{Struktur des Reaktionszeit-Spiels}
\includegraphics[width=\textwidth]{./../doc/images/Aufgabe2/Game_Struct_RTL (2).png}
\end{frame}
% -----------------------

% -----------------------
\begin{frame}[fragile]{Edge Detection}
\begin{columns}[T,onlytextwidth]
  \column{0.49\textwidth}
  \scriptsize
  \begin{lstlisting}[style=vhdl]
    entity EdgeDetection is
    port (
      iClk : in std_ulogic;
      inResetAsync : in std_ulogic;
      iEnable : in std_ulogic;
      iSync : in std_ulogic;
      oEdge : out std_ulogic);
  end EdgeDetection;
  \end{lstlisting}

  \column{0.49\textwidth}
  \scriptsize
  \begin{lstlisting}[style=vhdl]
architecture RTL of EdgeDetection is
  signal SyncPrev : std_ulogic;
begin

process (iClk, inResetAsync) is
  begin
    -- asynchronous reset
    if (inResetAsync = not('1')) then
      SyncPrev <= '0';

    elsif (rising_edge(iClk)) then
      if(iEnable = '1') then
        SyncPrev <= iSync;
      else
        SyncPrev <= SyncPrev;
      end if;
    end if;
end process;

-- combinational logic for edge detection
oEdge <= '1' when (iSync = '1' and SyncPrev = '0') else '0';

end architecture RTL;
  \end{lstlisting}
\end{columns}
\end{frame}
% -----------------------

% -----------------------
\begin{frame}{Edge Detection RTL Viewer}
\includegraphics[width=\textwidth]{./../doc/images/Aufgabe2/EdgeDetection_RTL.png}
\end{frame}
% -----------------------

% -----------------------
\begin{frame}{Finite State Machine  Moore}
\includegraphics[width= 11 cm]{./../doc/images/Aufgabe2/FSM.png}
\end{frame}
% -----------------------

% -----------------------
\begin{frame}[fragile]{State Register}
\begin{lstlisting}[style=vhdl]
  -- State Register
  process (iClk, inResetAsync) is
  begin
    if (inResetAsync = not('1')) then
      State <= Locked;
    elsif (rising_edge(iClk)) then
      if(iEnable = '0') then
        State <= State; -- hold state when not enabled
      else
        State <= NextState;
      end if;
    end if;
  end process;
\end{lstlisting}
\end{frame}
% -----------------------

% -----------------------
\begin{frame}[fragile]{Next State Logic}
\begin{columns}[T,onlytextwidth]
  \column{0.49\textwidth}
  \scriptsize
  \begin{lstlisting}[style=vhdl]
-- State Transition Process
NextStateLogic : process (State,iEnable,iA_Sync,iB_Sync) is
begin

  NextState <= State;    -- default hold state
  oLeds <= cLEDOFF;      -- default all leds off
  oZero <= '0';          -- default zero off
  oEnableCounter <= '0'; -- default counter disabled

  case State is
    when Locked => 
      if(iA_Sync = '1') then
        NextState <= Unlocked;
      end if;
      oZero <= '1';                    -- reset counters
      oLeds(cLED_LOCKED_INDEX) <= '1'; -- indicate locked state
      
    when Unlocked =>
      if(iB_Sync = '1') then
        NextState <= CountUpTime;
      end if;
      oLeds(cLED_UNLOCKED_INDEX) <= '1'; -- indicate locked state
  \end{lstlisting}

  \column{0.49\textwidth}
  \scriptsize
  \begin{lstlisting}[style=vhdl]
    when CountUpTime =>
      oEnableCounter <= '1'; -- enable counter
      if(iA_Sync = '1') then
        NextState <= ShowResult;
      end if;
      oLeds(cLED_COUNTUP_INDEX) <= '1'; -- indicate locked state
      
    when ShowResult =>
      if(iB_Sync = '1') then
        NextState <= Locked;
      end if;
      oLeds(cLED_SHOWRESULT_INDEX) <= '1'; -- indicate locked state
      
    when others =>
      NextState <= cStateAllOff;
  end case;

end process;
  \end{lstlisting}
\end{columns}
\end{frame}
% -----------------------

% -----------------------
\begin{frame}{Finite State Machine RTL-Viewer}
\includegraphics[width=\textwidth]{./../doc/images/Aufgabe2/FSM_RTL.png}
\end{frame}
% -----------------------

% -----------------------
\begin{frame}[plain]{Finite State Machine State Machine Viewer}
  % Hintergrundbild über die ganze Folie
  \begin{tikzpicture}[remember picture,overlay]
    \node at (current page.center) {
      \includegraphics[width=\paperwidth]{./../doc/images/Aufgabe2/StateMachineViewer (2).png}
    };

    % Zweites Bild unten rechts darüber
    \node[anchor=south east, xshift=-5mm, yshift=5mm]
      at (current page.south east) {
        \includegraphics[width=7cm]{./../doc/images/Aufgabe2/StateMachineViewer (1).png}
      };
  \end{tikzpicture}
\end{frame}
% -----------------------

% -----------------------
\begin{frame}[fragile]{Finite State Machine RTL-Viewer}
  \begin{columns}[T,onlytextwidth]
    \scriptsize
    \column{0.49\textwidth}
    \includegraphics[height=6cm]{./../doc/images/Aufgabe2/FSM_Ressource.png}
      \scriptsize
  \column{0.45\textwidth}
  \includegraphics[width=6cm]{./../doc/images/Aufgabe2/FSM_F_Max.png}
\end{columns}
\end{frame}
% -----------------------

% -----------------------
\begin{frame}[fragile]{Counter}
\begin{columns}[T,onlytextwidth]
  \column{0.49\textwidth}
  \scriptsize
  \begin{lstlisting}[style=vhdl]
entity Counter is
  generic(
    gCounterOverflowVal : natural := 10
  );
  port (
  iClk         : in  std_ulogic;
  iEnable      : in  std_ulogic;
  inResetAsync : in  std_ulogic;
  iZero        : in  std_ulogic;
  oOverflow    : out std_ulogic;
  oCount       : out unsigned(LogDualis(gCounterOverflowVal) downto 1));
end Counter;
  \end{lstlisting}

  \column{0.49\textwidth}
  \scriptsize
  \begin{lstlisting}[style=vhdl]
architecture RTL of Counter is
begin

  -- combinational logic for overflow output
  oOverflow <= '1' when (iEnable = '1' and iZero = '0' and oCount = gCounterOverflowVal - 1) else '0';

  process(iClk, inResetAsync) is
  begin
    if inResetAsync = not('1') then
      oCount    <= (others => '0');
    elsif rising_edge(iClk) then
      if iZero = '1' then
        oCount <= (others => '0');
      elsif iEnable = '1' then
        if oCount = gCounterOverflowVal - 1 then
          oCount    <= (others => '0');
        else
          oCount <= oCount + 1;
        end if;
      end if;
    end if;
  end process;
end architecture RTL;
  \end{lstlisting}
\end{columns}
\end{frame}
% -----------------------

% -----------------------
\begin{frame}{Counter RTL Viewer}
\includegraphics[width=\textwidth]{./../doc/images/Aufgabe2/Counter RTL.png}
\end{frame}
% -----------------------

% -----------------------
\begin{frame}[fragile]{Hex to 7-Segment Decoder}
\begin{columns}[T,onlytextwidth]
  \column{0.49\textwidth}
  \scriptsize
  \begin{lstlisting}[style=vhdl]
architecture Rtl of Hex2SevenSegment is

  function ToSevSeg(cValue : std_ulogic_vector(3 downto 0))
    return std_ulogic_vector is
  begin
    case cValue(3 downto 0) is
      when "0000" => return "0111111";
      when "0001" => return "0000110";
      when "0010" => return "1011011";
      when "0011" => return "1001111";
      when "0100" => return "1100110";
      when "0101" => return "1101101";
      when "0110" => return "1111101";
      when "0111" => return "0000111";
      when "1000" => return "1111111";
      when "1001" => return "1101111";
      when "1010" => return "1110111";
      when "1011" => return "1111100";
      when "1100" => return "0111001";
      when "1101" => return "1011110";
      when "1110" => return "1111001";
      when "1111" => return "1110001";
      when others => return "XXXXXXX";
    end case;
  end ToSevSeg;
  \end{lstlisting}

  \column{0.49\textwidth}
  \scriptsize
  \begin{lstlisting}[style=vhdl]
begin

  o7SegCode <= ToSevSeg(iHexValue);

end Rtl;
  \end{lstlisting}
\end{columns}

\end{frame}
% -----------------------

\begin{frame}{Simulation Reaction Time Game}
\includegraphics[width=\textwidth]{./../doc/images/Aufgabe2/GameWave (2).png}
\end{frame}
% -----------------------

\begin{frame}{Simulation Reaction Time Game}
\includegraphics[width=\textwidth]{./../doc/images/Aufgabe2/GameWave (1).png}
\end{frame}


\begin{frame}{PCB Adapter}
\includegraphics[width=\textwidth]{./../doc/images/Aufgabe2/PCB_Adapter_RTL_Cropped.png}
\end{frame}


% -----------------------
\begin{frame}{Ressource Summery}
  \begin{itemize}
    \item SyncStage: 2 x 2 Flip-Flops 
    \item EdgeDetection: 2 x 1 Flip-Flops 
    \item StrobeGen: 17 Flip-Flops 
    \item FSM:       4 Flip-Flops 
    \item Counter: 3 x 4 Flip-Flops 
  \end{itemize}
  Gesamt Anzahl Flip-Flops: 39
\end{frame}
% -----------------------

% -----------------------
\begin{frame}{Ressource Summery}
  \begin{columns}[T,onlytextwidth]
    \scriptsize
    \column{0.49\textwidth}
    \includegraphics[width=7cm]{./../doc/images/Aufgabe2/PCB_Adapter_Ressource.png}
      \scriptsize
  \column{0.45\textwidth}
  \includegraphics[width=6cm]{./../doc/images/Aufgabe2/PCB_Adapter_FMAX.png}
\end{columns}
\end{frame}
% -----------------------

% -----------------------
\begin{frame}{Test am DE1 SOC FPGA Board}

% ...

\href{./ReactionTimeGame.MOV}{\beamergotobutton{Video öffnen}}

\end{frame}
% -----------------------

% -----------------------
\begin{frame}[standout]
  Fragen?
\end{frame}

\end{document}
