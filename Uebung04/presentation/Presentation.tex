% ======================================================================
%  Code Review Presentation Template – CHD3 (University)
%  Author: Simon Offenberger
%  Date: \today
% ======================================================================

\documentclass[aspectratio=169,11pt]{beamer}

% =======================
%   THEMES & PACKAGES
% =======================
\usetheme{metropolis} % clean, professional theme
\usepackage[utf8]{inputenc}
\usepackage[T1]{fontenc}
\usepackage[ngerman]{babel}
\usepackage{graphicx}
\usepackage{booktabs}
\usepackage{listings}
\usepackage{xcolor}
\usepackage{pdfpages}


% =======================
%   DESIGN SETTINGS
% =======================
\definecolor{uniBlue}{RGB}{0,68,136}
\definecolor{lightGray}{RGB}{245,245,245}
\setbeamercolor{frametitle}{bg=uniBlue, fg=white}
\setbeamercolor{title}{fg=uniBlue}
\setbeamercolor{progress bar}{use=alerted text, fg=uniBlue}

% =======================
%   CODE LISTINGS STYLE
% =======================
\lstdefinestyle{codeStyle}{
    backgroundcolor=\color{lightGray},
    basicstyle=\ttfamily\small,
    keywordstyle=\color{uniBlue}\bfseries,
    commentstyle=\color{gray}\itshape,
    stringstyle=\color{orange},
    showstringspaces=false,
    frame=single,
    breaklines=true,
    tabsize=2
}
\lstset{style=codeStyle}

% =======================
%   TITLE INFORMATION
% =======================
\title[CHD3 Übung 4]{Übung 4, Aufgabe 3}
\subtitle{Analyse: Schaltplan des DE1 SOC Boards} 
\author{Simon Offenberger S2410306027@fhooe.at} 
\institute{FH Hagenberg}
\date{\today}

% =======================
%   FOOTLINE ANPASSUNG
% =======================
\setbeamertemplate{footline}{
  \leavevmode%
  \hbox{%
    \begin{beamercolorbox}[wd=.8\paperwidth,ht=2.5ex,dp=1.5ex,leftskip=1em]{author in head/foot}%
      \usebeamerfont{author in head/foot}\insertshortauthor
    \end{beamercolorbox}%
    \begin{beamercolorbox}[wd=.2\paperwidth,ht=2.5ex,dp=1.5ex,rightskip=1em plus1fil]{date in head/foot}%
      \usebeamerfont{date in head/foot}\insertframenumber{} / \inserttotalframenumber
    \end{beamercolorbox}%
  }%
  \vskip0pt%
}

\setbeamertemplate{navigation symbols}{} % entfernt die Navigationssymbole

% =======================
%   DOCUMENT
% =======================
\begin{document}

% -----------------------
\begin{frame}
  \titlepage
\end{frame}

% -----------------------
\begin{frame}{Agenda}
    \begin{itemize}
      \item Decoupling - Spannungsstabilisierung
      \item GPIO-Schutzbeschaltung
      \item IR-Emitter
  \end{itemize}
\end{frame}

% -----------------------
\begin{frame}{Decoupling - Spannungsstabilisierung}
    \includegraphics[width=15cm]{Images/Decoupling.png}
\end{frame}
% -----------------------
% -----------------------
\begin{frame}{Decoupling - Spannungsstabilisierung}
    \begin{columns}[T] % [T] = obere Ausrichtung
    \column{0.5\textwidth}
    \begin{itemize}
      \item Warum so viele Kondensatoren parallel?
      \begin{itemize}
        \item Power Pins sind räumlich getrennt.
      \end{itemize}
      \item Reicht hier ein großer Kondensator?
      \begin{itemize}
        \item höherer ESR (Equivalent Series Resistance)
        \item höhere ESL (Equivalent Series Inductance)
        \item längere Leitungen zu den Power Pins.
      \end{itemize}
    \end{itemize}
    \column{0.5\textwidth}
      \centering
      \includegraphics[height=7cm]{./../doc/images/Board.jpg}
  \end{columns}
\end{frame}
% -----------------------
% -----------------------
\begin{frame}{Decoupling - Spannungsstabilisierung}
      \includegraphics[width=15cm]{Images/Schematik_Decoupling.png}
\end{frame}
% -----------------------
% -----------------------
\begin{frame}{GPIO-Schutzbeschaltung}
    \begin{columns}[T] % [T] = obere Ausrichtung
    \column{0.3\textwidth}
      \includegraphics[height=7cm]{./../doc/images/Schaltplan_Dioden.png}
    \column{0.7\textwidth}
      \centering
      \includegraphics[width = 10cm]{./../doc/images/Schaltpan_R.png}
  \end{columns}
\end{frame}
% -----------------------
% -----------------------
\begin{frame}{GPIO-Schutzbeschaltung Schutz vor Über- bzw. negativer Spannung}
    \begin{columns}[T] % [T] = obere Ausrichtung
    \column{0.5\textwidth}
      \includegraphics[width=8cm]{./Images/GPIO_Schutz_pos.png}
    \column{0.5\textwidth}
      \centering
      \includegraphics[width=8cm]{./Images/GPIO_Schutz_neg.png}
  \end{columns}
\end{frame}
% -----------------------
% -----------------------
\begin{frame}{GPIO-Schutzbeschaltung Schutz vor Überstrom}
      \includegraphics[width=8cm]{./Images/GPIO_Schutz_Kurzschluss.png}
      \[
      I_k = \dfrac{3.3\,\text{V}}{47\,\Omega} = 70mA
      \]
      \\
      Absolute Maximum Pin Current : -25mA / 40mA
\end{frame}
% -----------------------
% -----------------------
\begin{frame}{IR-Emitter Leistung an den Widerständen}
    \begin{columns}[T] % [T] = obere Ausrichtung
    \column{0.5\textwidth}
\begin{align*}
I &= \frac{V_{CC3P3} - V_f - V_{CE}}{R} \\
  &= \frac{3.3\,\mathrm{V} - 1.7\,\mathrm{V} - 0.2\,\mathrm{V}}{2.5\,\Omega} \\
  &= 560\,\mathrm{mA}
\end{align*}

\begin{align*}
P_R &= R \cdot I^2\\
    &= 2.5\,\Omega \times (0.56\,\mathrm{A})^2 \\
    &= 0.784\,\mathrm{W}
\end{align*}
    $P_{R190} = P_{R191} = \frac{P_R}{2} = 0.392\,\mathrm{W}$
      \column{0.5\textwidth}
      \includegraphics[width=8cm]{./Images/IR-Emitter.png}
  \end{columns}
\end{frame}
% -----------------------
% -----------------------
\begin{frame}{IR-Emitter Leistung an den Widerständen}
  \begin{itemize}
    \item Darf die LED dauerhaft eingeschaltet sein?
  \end{itemize}
  \begin{table}[h!]
\centering
\caption{Typische Nennleistungen von SMD-Widerständen}
\begin{tabular}{|c|c|c|}
\hline
\textbf{Baugröße (imperial)} & \textbf{Nennleistung [W]} \\ \hline
0201  & 0.05 \\ \hline
0402  & 0.063 \\ \hline
0603  & 0.10 \\ \hline
0805  & 0.125 \\ \hline
1206  & 0.25 \\ \hline
1210  & 0.33 \\ \hline
2010  & 0.50 \\ \hline
2512  & 1.00 \\ \hline
\end{tabular}
\end{table}
\end{frame}
% -----------------------
% -----------------------
\begin{frame}{IR-Emitter Leistung an der LED}
  $P_{D} = V_f \cdot I = 1.7\,\mathrm{V} \times 560\,\mathrm{mA} = 0.952\,\mathrm{W}$
  \begin{itemize}
    \item Warum überhitzt die LED nicht?
    \item pulsartige Ansteuerung -> meist für Datenübertragung
    \item Wirkungsgrad von etwa 30\% -> Verlustleistung ca. 0.7W
    \item Effiziente Wärmeableitung über Kupferfläche
  \end{itemize}
\end{frame}
% -----------------------
% -----------------------
\begin{frame}{Spannungsversorgung - Verpolungsschutz}
\includegraphics[width=13cm]{Images/ReversePolarity.png}
\end{frame}
% -----------------------
% -----------------------
\begin{frame}{Spannungsversorgung}
\includegraphics[width=13cm]{Images/Switching Powersupply.png}
\end{frame}
% -----------------------

\includepdf[pages=28-30]{C:/HSD_SEM_3/CHD3/Uebung04/literature/terasic/DE1-SoC/DE1-SoC.pdf}

% -----------------------
\begin{frame}{Spannungsebenen am Board }
\begin{itemize}
  \item 1.1V / 8A
  \item 3.3V / 5A
  \item 2.5V / 3A
  \item 1.5V / 3A
  \item 9V / 0.5A
  \item 1.2V / 1.1A
  \item 1.8V / 1.1A
  \item 1.5V -> DDR3
\end{itemize}
\end{frame}
% -----------------------
\begin{frame}[standout]
  Fragen?
\end{frame}

\end{document}
